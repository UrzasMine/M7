\section{Theoretische Grundlagen}

\subsection{Kontinuitätsgleichung}

Wenn eine Flüssigkeit oder Gas mit nicht veränderlichen Dichte durch ein Rohr fließt welches einen veränderlichen Querschnitt hat, dann gilt die Kontinuitätsgleichung,
\begin{equation}
    \centering
    v_1 \cdot A_1 = v_2 \cdot A_2 
    \label{eq:Kontinuität}
\end{equation}
dementspechend verhält sich die Strömungsgeschwindigkeit indirekt proportional zu der Querschnittsfläche\cite{Tipler}.

\subsection{Bernoulli-Gleichung}

Innerhalb eines Rohres lässt sich der gesamt Druck der Flüssigkeit oder des Gases mithilfe der Bernoulli-Gleichung
\begin{equation}
    \centering
    \rho \cdot g \cdot h + \frac{\rho}{2} v^2 + p = const.
    \label{eq:Bernoulli}
\end{equation}
berechnen\cite{Tipler}. Hierbei beschreibt der erste Teil der Gleichung den Höhendruck,

