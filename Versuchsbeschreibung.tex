\section{Versuchsbeschreibung}

Der Versuch wird mithilfe eines Windkanals durchgeführt. Zur Messung der gesuchten Werte wird ein Prandtl-Rohr für die Strömungsgeschwindigkeit sowie zwei Newtonmeter für die Reibungs- und die Gewichtskraft verwendet.

\section{Versuchsaufbau und Durchführung}

\subsection{Kontinuitätsgleichung}

Zum Nachweis der Kontinuitätsgleichung (Gleichung~1.1) wird im Windkanal eine
Bernoulli-Rampe installiert. An dieser Rampe befindet sich eine Skalierung, welche
die jeweilige Querschnittsfläche des Kanals angibt. Durch den sich verringernden
Querschnitt entlang der Rampe erhöht sich die Strömungsgeschwindigkeit.

Am Windgenerator wird eine konstante Windgeschwindigkeit eingestellt.
An verschiedenen Positionen entlang der Rampe wird mithilfe eines Prandtl-Rohres
der Differenzdruck zwischen Gesamtdruck und statischem Druck gemessen.
Dabei wird darauf geachtet, dass sich das Druckmessgerät mittig im Kanal und
parallel zur Strömungsrichtung befindet.

Abbildung \ref{konti} zeigt den Versuchsaufbau zum Nachweis der Kontinuitätsgleichung.

\begin{figure}[h]
    \centering
    \includegraphics[width=\linewidth]{Bilder/teil1.PNG}
    \caption{Aufbau zum Nachweis der Kontinuitätsgleichung \cite{Protokkoll}}
    \label{konti}
\end{figure}

\subsection{Gleitzahl des Tragflügels}

Im zweiten Teil des Versuchs wird die Gleitzahl eines Tragflügels bestimmt.
Hierzu wird die Bernoulli-Rampe entfernt und durch einen ebenen Boden ersetzt.
Der Tragflügel wird an einer beweglichen Vorrichtung befestigt, die sowohl eine
Verstellung des Anstellwinkels als auch die Messung der Auftriebskraft ermöglicht.
Die Vorrichtung wird mittig im Windkanal positioniert und über ein Seil mit einem
Widerstandskraftmesser verbunden.

Der Versuch beginnt mit einem Anstellwinkel von $14^\circ$. Die Windgeschwindigkeit
wird so eingestellt, dass die gemessene Widerstandskraft innerhalb des Messbereichs
liegt und während dieses Versuchsteils konstant bleibt. Zur Bestimmung der
Strömungsgeschwindigkeit wird erneut mit einem Prandtl-Rohr der Druck gemessen.
Anschließend können die Auftriebs- und Widerstandskraft an den entsprechenden
Kraftmessern abgelesen werden.

Der Anstellwinkel wird schrittweise um $2^\circ$ verringert, wobei die Messungen
für jeden Winkel wiederholt werden. Dieses Vorgehen wird bis zu einem Anstellwinkel
von $-8^\circ$ durchgeführt. Der Versuchsaufbau ist in Abbildung \ref{gleit} dargestellt.

\begin{figure}[h]
    \centering
        \includegraphics[width=\linewidth]{Bilder/teil2.PNG}

    \caption{Aufbau zur Bestimmung der Gleitzahl \cite{Protokkoll}}
    \label{gleit}
\end{figure}

\subsection{Widerstandsbeiwerte}

Zur Bestimmung der Widerstandsbeiwerte verschiedener geometrischer Körper werden
zunächst deren charakteristische Abmessungen, wie Durchmesser beziehungsweise
Länge und Breite, gemessen. Anschließend wird die Windgeschwindigkeit im Windkanal
eingestellt. Der zur Geschwindigkeitsbestimmung notwendige Druck wird gemessen
und während der Messreihe konstant gehalten.

Die zu untersuchenden Körper werden nacheinander an der Halterung im Windkanal
befestigt und die jeweils auftretende Widerstandskraft wird gemessen.
