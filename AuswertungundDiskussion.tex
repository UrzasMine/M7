\section{Auswertung}

\subsection{Kontinuitätsgleichung}

Die im ersten Teil des Versuches ermittelten Werte wurden in Tabelle \ref{tb:1} dargestellt, sowie der hydrostatische Druck und die mithilfe von Gleichung (\ref{strö}) ausgerechneten Strömungsgeschwindigkeiten. Der Fehler dieser Strömungsgeschwindigkeiten beträgt
\begin{equation}
    \centering
    \Delta v = \pm \left(\left|\frac{\partial v}{\partial p_\text{H}}\right|\cdot \Delta p_\text{H}\right) = \pm \frac{\Delta p_\text{H}}{\sqrt{2 \cdot \rho \cdot p_\text{H}}}.
    \label{eq:Fehlerv}
\end{equation}
Zusätzlich konnte die Drucksonde nicht perfekt auf den verwendeten Flächenabschnitt eingestellt werden, weswegen dort ein abgeschätzter Fehler von $\pm 2~\text{cm}^2$ angenommen wird. Zudem wurde die Raumtemperatur auf 20°C geschätzt, wodurch die Luftdichte $\rho = 1,204~\text{kg/m}^3$ (Quelle) beträgt.  
Andere Fehlerquellen sind Ablesefehler bei Drücken und Geschwindigkeiten, da die Skala nicht linear ansteigt. Aufgrund dessen wird der Fehler beim Manometer als $\Delta p = \pm 3~\text{Pa}$ und $\Delta v = \pm 1~\text{m/s}$ geschätzt.

\begin{table}[H]
    \centering
    \begin{tabular}{c|c|c}
        Fläche $A/\text{m}^2$ & Hydrostatischer Druck $p_H/\text{Pa}$ & Geschwindigkeit $v/\text{m/s}$\\
        \hline
        $0,020$ & $5,2$ & $23$ \\
        $0,019$ & $5,3$ & $24$ \\
        $0,018$ & $5,5$ & $25$ \\
        $0,017$ & $5,7$ & $27$ \\
        $0,016$ & $5,9$ & $29$ \\
        $0,015$ & $6,2$ & $32$ \\
    \end{tabular}
    \caption{Gemessene Werte aus dem ersten Teil des Versuchs}
    \label{tb:1}
\end{table}

Aus diesen Werten lässt sich nun $a := A \cdot v$ berechnen. Der Fehler für $a$ beträgt
\begin{equation}
    \centering
    \Delta a = \pm \left(\left|\frac{\partial a}{\partial A}\right|\cdot \Delta A + \left|\frac{\partial a}{\partial v}\right|\cdot \Delta v \right) = \pm \left(\Delta A \cdot v + A \cdot \Delta v\right)
    \label{Fehler-a}
\end{equation}

Wenn nun diese Werte für $a$ gegen die Oberfläche $A$ aufgetragen werden, ergibt sich der Graph \ref{if:AV}. In diesem Graphen ist zu erkennen, dass die Werte recht gut innerhalb des Fehlers liegen. 
\begin{figure}[H]
    \centering
    \includegraphics[width=\linewidth]{Bilder/AV.png}
    \label{if:AV}
\end{figure}

In diesem Graphen ist nun eine positive Steigung zu erkennen, obwohl es eigentlich keine Steigung geben sollte, da dieser Wert konstant sein sollte. Gründe für diese Steigung könnten sein, dass die Drucksonde im Versuchsaufbau beweglich ist oder ein Spalt im oberen Teil des Windkanals vorhanden ist, wodurch ein Druckverlust entsteht. Dieser Druckverlust würde jedoch eher zu einer negativen Steigung im Graphen führen. Weitere Fehlerquellen sind, dass das Prandtl-Rohr parallel zur Windrichtung ausgerichtet werden muss, was nicht perfekt möglich war, und dass die Drucksonde schwer auf der perfekten Höhe zum Messen gehalten werden konnte. Einer der wahrscheinlich größten Fehler kommt von der großen Schrittweite der Druckskala, welche zu einem sehr hohen statistischen Fehler führen kann.

\subsection{Widerstands- und Auftriebskräfte}

\begin{table}[H]
    \centering
    \begin{tabular}{c|c|c}
        Winkelstellung / ° & Auftriebskraft $F_W$ / N & Widerstandskraft $F_A$ / N \\
        \hline
        $14 \pm 1$ & $0,585 \pm 0,1$ & $2,15 \pm 0,15$ \\
        $12 \pm 1$ & $0,49 \pm 0,1$ & $2,05 \pm 0,15$ \\
        $10 \pm 1$ & $0,46 \pm 0,1$ & $2,0 \pm 0,15$ \\
        $8 \pm 1$ & $0,42 \pm 0,1$ & $1,85 \pm 0,15$ \\
        $6 \pm 1$ & $0,34 \pm 0,1$ & $1,65 \pm 0,15$ \\
        $4 \pm 1$ & $0,27 \pm 0,1$ & $1,45 \pm 0,15$ \\
        $2 \pm 1$ & $0,22 \pm 0,1$ & $1,15 \pm 0,15$ \\
        $0 \pm 1$ & $0,18 \pm 0,1$ & $1,1 \pm 0,15$ \\
        $-2 \pm 1$ & $0,12 \pm 0,1$ & $0,6 \pm 0,15$ \\
        $-4 \pm 1$ & $0,1 \pm 0,1$ & $0,5 \pm 0,15$ \\
        $-6 \pm 1$ & $0,09 \pm 0,1$ & $0 \pm 0,15$ \\
        $-8 \pm 1$ & $0,065 \pm 0,1$ & $-0,3 \pm 0,15$ \\
    \end{tabular}
    \caption{Werte aus dem zweiten Experiment mit gemessenen Winkeleinstellungen von 14° bis -8°}
    \label{tb:2}
\end{table}

Die in Teil zwei des Experiments gemessenen Werte wurden in Tabelle \ref{tb:2} dargestellt. Um aus den Werten der Tabelle die Widerstands- und Auftriebswerte zu berechnen, wird angenommen, dass die angeströmte Fläche $S$ aus den Gleichungen $F_A$ und $F_W$ gleich der gesamten Oberfläche des Tragflügels ist. Diese ist gegeben durch
\begin{equation}
    \centering
    S = B \cdot L
    \label{Flügel}
\end{equation}
Der Fehler der Tragfläche ist durch
\begin{equation}
    \centering
    \Delta S = \pm \left(\left|\frac{\partial S}{\partial L}\right|\cdot \Delta L + \left|\frac{\partial S}{\partial B}\right|\cdot \Delta B\right) = \pm \left(\Delta L \cdot B + L \cdot \Delta B\right)
    \label{Fehler-Fläche}
\end{equation}
gegeben. Mit $B = 14 \pm 0,3~\text{cm}$ und $L = 22 \pm 0,3~\text{cm}$ beträgt die Tragfläche des Flügels $S = 308~\text{cm}^2$.  

Um den Luftwiderstandsbeiwert $c_W$ zu berechnen, wird Gleichung $F_W$ zu
\begin{equation}
    \centering 
    c_W = \frac{2 \cdot F_W}{\rho \cdot v^2 \cdot S}
    \label{Cw}
\end{equation}
umgeformt. Der Fehler von $c_W$ ist durch
\begin{equation}
    \centering
    \Delta c_W = \pm \left(\left|\frac{\partial c_W}{\partial F_W}\right| \cdot \Delta F_W + \left|\frac{\partial c_W}{\partial v}\right| \cdot \Delta v + \left|\frac{\partial c_W}{\partial S}\right| \cdot \Delta S\right)
\end{equation}
gegeben. Nach dem Ausdifferenzieren ergibt dies
\begin{equation}
    \centering
    \Delta c_W = \pm \left(\frac{2 \cdot \Delta F_W}{\rho \cdot v^2 \cdot S} + \frac{4 \cdot F_W \cdot \Delta v}{\rho \cdot v^3 \cdot S} + \frac{2 \cdot F_W \cdot \Delta S}{\rho \cdot v^2 \cdot S^2}\right)
    \label{eq:FcW}
\end{equation}

Werden nun die Auftriebs- und Widerstandskräfte in einem Diagramm aufgetragen, kann der ideale Anstellwinkel ermittelt werden. Dazu wird eine Ursprungsgerade von Hand in das Diagramm gelegt, so dass diese den Graphen der Messpunkte berührt. Dies ist in der nächsten Abbildung zu erkennen.
\begin{figure}[h]
    \centering
    \includegraphics[width=\linewidth]{Bilder/epsilon.png}
    \caption{Der Auftriebsbeiwert aufgetragen gegen den Widerstandsbeiwert. Die Ausgleichsgeraden wurden mittels \texttt{numpy.polyfit} erstellt. Es wurde eine Funktion der Form $a \cdot \ln(b \cdot x + c) + d$ verwendet. Die Steigung der Tangente wurde mittels Augenmaß bestimmt.}
    \label{optiwinkel}
\end{figure}

Anhand dieser Grafik kann nun der Berührungspunkt abgelesen werden; dieser befindet sich bei ca. 0°. Damit ergibt sich als Gleitzahl
\begin{equation}
    \centering
    \epsilon = 5,67
\end{equation}

Mithilfe der Gleitzahl kann eine Aussage zum Verhältnis zwischen Auftriebskraft und Widerstandskraft getroffen werden: Je geringer die Gleitzahl, desto idealer ist das Verhältnis zwischen Auftrieb und Reibung. Um den Flügel effizient verwenden zu können, muss der ideale Anstellwinkel gefunden werden, bei dem die Gleitzahl so gering wie möglich ist. Dieser Winkel liegt in dieser Durchführung bei 0°; hierbei wird von einem Fehler von $\pm 2°$ ausgegangen.  

Das Ergebnis von 0° ist jedoch nicht realistisch, da die Auftriebskraft hier 0~N betragen sollte. Auch bei negativen Anstellwinkeln wurde noch Auftriebskraft gemessen. Hier ist von einem großen systematischen Fehler auszugehen. Dieser relativ große Fehler hat mehrere Ursachen: Zum einen konnte die Gerade durch den Ursprung nicht perfekt auf den Graphen gelegt werden, und zum anderen konnte der Berührungspunkt nicht exakt abgelesen werden. Weitere Fehlerquellen sind, dass die Tragfläche während des Versuchs gegen die Wände des Windkanals gestoßen ist, wodurch Ungenauigkeiten beim Ablesen entstanden. Weitere Unsicherheiten ergeben sich durch empfindliche Winkeleinstellungen, kleine Veränderungen der Höhe führen bereits zu großen Unterschieden im Winkel. Zusätzlich konnten die Kraftmesser vor dem Experiment nicht kalibriert werden, wodurch bereits Anfangswerte in den Fehler einfließen.

\subsection{Verschiedene Körper im Windkanal}

Im dritten Teil des Versuches wird nun für mehrere Verschiedene Körper die Widerstandskraft $c_\text{W} $ ermittelt. Hierzu wird Formel \ref{Widerstandskraft} verwendet und nach $c_\text{W}$ umgestellt. Für alle Körper außer dem Auto wird für A die Formel des Kreises verwendet, der Fehler für diese ist durch
\begin{equation}
\centering
\Delta A = \pm \left|\frac{ \partial A} { \partial r} \right| \cdot \Delta r = \pm 2\pi \cdot r \cdot \Delta r
\end{equation}
gegeben. Für die Oberfläche des Fahrzeugs wird angenommen dass es sich um ein Rechteck handelt, dadurch ergibt sich für den Fehler
\begin{equation}
\centering
\Delta A = \pm \left(\left|\frac{\partial A}{\partial L}\right| \cdot \Delta X + \left|\frac{\partial A}{\partial B}\right| \cdot \Delta X\right) = \pm \left(L + B\right) \Delta X
\end{equation}
Der Fehler für $c_\text{W}$ ist durch
\begin{equation}
    \centering
\Delta c_W
=
\pm \left(
\left| \frac{\partial c_\text{W}}{\partial F_W} \right| \Delta F_W
+
\left| \frac{\partial c_W}{\partial p_d} \right| \Delta p_d
+
\left| \frac{\partial c_W}{\partial A} \right| \Delta A
\right)
 = 
\Delta c_W
=
\pm \left(
\frac{\Delta F_W}{p_d\,A}
+
\frac{F_W\,\Delta p_d}{p_d^{2}\,A}
+
\frac{F_W\,\Delta A}{p_d\,A^{2}}
\right)
\end{equation}
gegeben.
Die Tabelle listet die Widerstandsbeiwerte im Vergleich zu den Literatur Werten \cite{Tipler} auf.

\begin{table}[H]
\centering
\begin{tabular}{c|c|c|c|c}
    Körper & Fläche A $\left[cm^2\right]$ & Kraft $F_\text{W}$ $\left[N\right]$ & $c_\text{W}$ & $c_\text{W, Literatur}$ \\
\hline
Scheibe  & 12,6 $\pm 1,2$ & 0,07 $\pm 0,05$ N & 0,40& 1,2\\ 
Scheibe & 24,6 $\pm 1,2$ & 0,13$\pm 0,05$ N & 0,33 & 1,2\\
Scheibe & 50,3 $\pm 1,2$ & 0,305$\pm 0,05$ N & 0,34 & 1,2\\ 
Tropfen $\left(\text{Spitz}\right)$ & $24,6 \pm 3,7$ &  0,04 $\pm 0,05$ N& 0,15\\
Tropfen $\left(\text{Rund}\right)$ & $24,6 \pm 3,7$& 0,02 $\pm 0,05$ N& 0,10 & 0,06\\ 
Ball & 24,6 $\pm 3,7$ & 0,06$\pm 0,05$ N & 0,18 & 0,4\\ 
Halbkugel & 24,6 $\pm 3,7$ & 0,18 $\pm 0,05$ N& 0,12 & 0,8\\

\caption{Die Flächen Widerstandkräfte und Wiederstandeiwerte der verschiedenen Formen bei konstanter windgeschwindigkeit}

\end{tabular}
    
\end{table}

\subsection{quadratische Abhängigkeit des Luftwiderstandes}
Trägt man den Luftwiderstand logarithmisch gegen die Windgeschwindigkeit auf so lässt sich ein linearer zusammenhang wie in Abbildung \ref{lnauftragung} erkennen. Dadurch lässt sich der $c_\text{W} = 0,576$ Wert deutlich besser bestimmen der relative fehler zum literautwert beträgt jetzt "nur" noch 40 \% wohingegn bei der Vorherrigen Aufgabe der fehler 660 \% betrug. Das die Messung hier besser funktioniert hat kann auf die Mehrzahl an Messpunkten zurückgeführ werden. 

\begin{figure}[H]
    \centering
        \includegraphics[width=\linewidth]{Bilder/lnauftraggung.png}

    \caption{Der Luftwiderstand logarithmisch aufgetragen gegen die Windgeschwindigkeit. Die Ausgleichsgeraden wurden mittels \texttt{numpy.polyfit} erstellt.}
    \label{lnauftragung}

    
\end{figure}

\subsection{Verschiedene Körper im Windkanal}

Im dritten Teil des Versuches wird nun für mehrere verschiedene Körper die Widerstandskraft $c_\text{W}$ ermittelt. Hierzu wird Formel \ref{Widerstandskraft} verwendet und nach $c_\text{W}$ umgestellt. Für alle Körper außer dem Auto wird für $A$ die Formel des Kreises verwendet, der Fehler für diese ist durch
\begin{equation}
\centering
\Delta A = \pm \left|\frac{\partial A}{\partial r} \right| \cdot \Delta r = \pm 2 \pi \cdot r \cdot \Delta r
\end{equation}
gegeben. Für die Oberfläche des Fahrzeugs wird angenommen, dass es sich um ein Rechteck handelt, dadurch ergibt sich für den Fehler
\begin{equation}
\centering
\Delta A = \pm \left(\left|\frac{\partial A}{\partial L}\right| \cdot \Delta X + \left|\frac{\partial A}{\partial B}\right| \cdot \Delta X\right) = \pm (L + B) \Delta X
\end{equation}
Der Fehler für $c_\text{W}$ ist durch
\begin{equation}
    \centering
\Delta c_W
=
\pm \left(
\left| \frac{\partial c_\text{W}}{\partial F_W} \right| \Delta F_W
+
\left| \frac{\partial c_W}{\partial p_d} \right| \Delta p_d
+
\left| \frac{\partial c_W}{\partial A} \right| \Delta A
\right)
 = 
\Delta c_W
=
\pm \left(
\frac{\Delta F_W}{p_d\,A}
+
\frac{F_W\,\Delta p_d}{p_d^{2}\,A}
+
\frac{F_W\,\Delta A}{p_d\,A^{2}}
\right)
\end{equation}
gegeben.  
Die Tabelle listet die Widerstandsbeiwerte im Vergleich zu den Literaturwerten \cite{Tipler} auf.

\begin{table}[H]
\centering
\begin{tabular}{c|c|c|c|c}
    Körper & Fläche $A$ [cm$^2$] & Kraft $F_\text{W}$ [N] & $c_\text{W}$ & $c_\text{W, Literatur}$ \\
\hline
Scheibe  & 12,6 $\pm 1,2$ & 0,07 $\pm 0,05$ & 0,40 & 1,2 \\ 
Scheibe & 24,6 $\pm 1,2$ & 0,13 $\pm 0,05$ & 0,33 & 1,2 \\
Scheibe & 50,3 $\pm 1,2$ & 0,305 $\pm 0,05$ & 0,34 & 1,2 \\ 
Tropfen (Spitz) & 24,6 $\pm 3,7$ & 0,04 $\pm 0,05$ & 0,15 & - \\
Tropfen (Rund) & 24,6 $\pm 3,7$ & 0,02 $\pm 0,05$ & 0,10 & 0,06 \\ 
Ball & 24,6 $\pm 3,7$ & 0,06 $\pm 0,05$ & 0,18 & 0,4 \\ 
Halbkugel & 24,6 $\pm 3,7$ & 0,18 $\pm 0,05$ & 0,12 & 0,8 \\
\end{tabular}
\caption{Die Flächen, Widerstandskräfte und Widerstandsbeiwerte der verschiedenen Formen bei konstanter Windgeschwindigkeit}
\end{table}

\subsection{Quadratische Abhängigkeit des Luftwiderstandes}

Trägt man den Luftwiderstand logarithmisch gegen die Windgeschwindigkeit auf, so lässt sich ein linearer Zusammenhang wie in Abbildung \ref{lnauftragung} erkennen. Dadurch lässt sich der $c_\text{W} = 0,576$ Wert deutlich besser bestimmen. Der relative Fehler zum Literaturwert beträgt jetzt „nur“ noch 40\%, wohingegen bei der vorherigen Aufgabe der Fehler 660\% betrug. Dass die Messung hier besser funktioniert hat, kann auf die Mehrzahl an Messpunkten zurückgeführt werden. 

\begin{figure}[H]
    \centering
    \includegraphics[width=\linewidth]{Bilder/lnauftraggung.png}
    \caption{Der Luftwiderstand logarithmisch aufgetragen gegen die Windgeschwindigkeit. Die Ausgleichsgeraden wurden mittels \texttt{numpy.polyfit} erstellt.}
    \label{lnauftragung}
\end{figure}
