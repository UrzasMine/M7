\section{Zusammenfassung}

\section*{Versuchsauswertung}

Eines der zentralen Ziele des Versuches, die Bestätigung der Kontinuitätsgleichung, konnte erfolgreich durchgeführt werden. Die Messung der Auftriebs- und Widerstandskräfte des Flügels stellte sich jedoch als nicht einfach heraus. Vor allem wurden unrealistische Auftriebskräfte für negative Anstellwinkel gemessen. Dies könnte durch ein genaueres Newtonmeter bzw. eine bessere Aufhängung gelöst werden. Zudem begann der Flügel bei höheren Windgeschwindigkeiten unkontrolliert zu wackeln, was ebenfalls die Messung erschwerte. 

Als Nächstes wurden Widerstandsbeiwerte verschiedener Formen gemessen. Auch wenn die Ergebnisse teilweise stark von den Literaturwerten abwichen, konnten die relativ „windschnittigeren“ Formen wie in der Literaturvorlage identifiziert werden. Als am „windschnittigsten“ hat sich hierbei der Tropfen herausgestellt. 

Abschließend wurde der quadratische Zusammenhang zwischen Windgeschwindigkeit und Luftwiderstand bestätigt. Hier konnte zudem der \(c_\text{W}\)-Wert der Halbkugel besser bestimmt werden.
