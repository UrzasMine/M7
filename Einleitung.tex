\section{Einleitung}
Der Windkanal ist ein wichtiger Experimentierort für alle möglichen Anwendungen in der Grundlagenforschung, aber auch in der Weiterentwicklung von Technik. Aufgrund dessen werden in diesem Versuch die Grundlagen für die Benutzung des Windkanals vermittelt, diese Grundlagen werden anhand von simplen, aber wichtigen Ergebnissen nähergebracht. Der Windkanal findet nicht nur in der Grundlagenforschung Anwendung, sondern wird auch gezielt in der Luft- und Raumfahrttechnik eingesetzt, um die Aerodynamik von Flugzeugen, Hubschraubern oder Drohnen zu analysieren und zu optimieren. Ebenso dient er in der Automobilindustrie dazu, den Luftwiderstand von Fahrzeugen zu reduzieren und den Kraftstoffverbrauch zu verbessern. Darüber hinaus wird der Windkanal in der Bau- und Architekturtechnik genutzt, um die Auswirkungen von Wind auf Gebäude, Brücken oder Windkraftanlagen zu untersuchen.  Der Windkanal ist im Bereich der Mechanik anzufinden und verwendet als Hauptbestandteil eine Turbine mit einem angebauten Kanal, in dem einige Experimente aufgebaut werden können. Für diese Experimente kommen Newtonmeter zum Einsatz, um die Auf- und Wiederstandskräfte messen zu können. Außerdem kommt ein Prandtl-Rohr zum Einsatz, welches dazu verwendet werden kann, Staudruck zu messen und somit die Windgeschwindigkeit zu ermitteln.
