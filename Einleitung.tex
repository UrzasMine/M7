\section{Einleitung}
Der Windkanal ist ein wichtiger Experimentierort für alle möglichen Anwendungen in der Grundlagenforschung, aber auch in der Weiterentwicklung von Technik. Aufgrund dessen werden in diesem Versuch die Grundlagen für die Benutzung des Windkanals vermittelt, diese Grundlagen werden anhand von simplen, aber wichtigen Ergebnissen nähergebracht. Der Windkanal ist im Bereich der Mechanik anzufinden und verwendet als Hauptbestandteil eine Turbine mit einem angebauten Kanal, in dem einige Experimente aufgebaut werden können. Für diese Experimente kommen Newtonmeter zum Einsatz, um Kräfte messen zu können. Außerdem kommt ein Prandtl-Rohr zum Einsatz, welches dazu verwendet werden kann, Luftdruck zu messen und somit die Windgeschwindigkeit zu ermitteln.
